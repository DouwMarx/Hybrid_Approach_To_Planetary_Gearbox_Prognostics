\documentclass[]{article}
\usepackage{graphicx}
\usepackage{float}
%\usepackage{natbib}
\usepackage{booktabs}
%opening
\title{Lumped mass model selection for use in a hybrid approach to prognostics in a planetary gearbox}
\author{}

\usepackage[utf8]{inputenc}
\usepackage[english]{babel}

%Import the natbib package and sets a bibliography  and citation styles
\usepackage{natbib}
\bibliographystyle{abbrvnat}


\begin{document}
	
	\maketitle
	
	%\begin{abstract}
	
	%\end{abstract}
	A simple lumped mass model is required for a planetary gearbox with a single planet gear.
	
	\section*{Problem introduction}
	
	This research problem concerns a hybrid prognostics method for crack length prediction in planetary gearboxes. The word "hybrid" in this context refers to the combination of physics-based and data-driven models in order to exploit the benefits of the respective approaches.
	
	\begin{itemize}
		\item The cracks are initiated in the root of a planet gear tooth and grown in an external fatigue test rig. The cracked gears can be inserted into the gearbox without gearbox disassembly.
		\item The crack length in the gear tooth will be inferred from accelerometer readings.
		\item A single planet gear will be used in the gearbox to simplify the problem.
	\end{itemize}
	
	An appropriate lumped mass model is required for use in the physics-based component of the hybrid prognostics strategy. A finite element model is used to compute the time-varying mesh stiffness.
	
	
	\section*{Experimental setup}
	Figure \ref{f:overall} shows the experimental setup. The gearbox under investigation (2) (speed up)  is driven by another gearbox (3) (speed down), used to increase the input torque. A hydraulic pump (1) is used to apply a load to the gearbox.
	
	
	\begin{figure}[H]
		\hspace*{-2cm} 
		\includegraphics[scale=0.12]{overall.pdf}
		\caption{Experimental setup}
		\label{f:overall}
	\end{figure}
	
	The specifications for the gearbox are listed in Table \ref{t:Bonfiglioli}.
	
	\begin{table}[H]
		\centering
		\caption{Bonfiglioli 300-L 1-5.77-PC-V01 B·E specifications}
		\label{t:Bonfiglioli}
		\begin{tabular}{@{}lll@{}}
			\toprule
			Symbol & Meaning & Value \\ \midrule
			$z_r$ & Number of ring gear teeth & $62$ \\
			$z_s$ & Number of sun gear teeth & $13$ \\
			$z_p$ & Number of planet gear teeth & $24$ \\
			&  &  \\
			$N$ & Number of planet gears & $3$ \\
			&  &  \\
			$R$& Gear Ratio  & $5.77$  \\\bottomrule
		\end{tabular}
	\end{table}
	
	\vspace{1cm}
	
	Figure \ref{f:internals} shows the planet carrier with planet gears inside the gearbox housing.
	
	\begin{figure}[H]
		\includegraphics[width=\linewidth, angle=180]{internals.jpg}
		\caption{Planet carrier and planet gears}
		\label{f:internals}
	\end{figure}
	
	
	
	Figure \ref{f:inspect} shows that the planet gear is accessible for removal in order to further grow a tooth root crack in an external fatigue rig.
	
	\begin{figure}[H]
		\includegraphics[width=\linewidth]{front.jpg}
		\caption{Inspection hole}
		\label{f:inspect}
	\end{figure}
	
	
	\section*{Choice of lumped mass model}
	
	An appropriate lumped mass model should be selected to be used in the prognostics framework.
	
	Given that that main focus of the research project is the hybrid methodology, a simple, easy to solve lumped mass model would be ideal. An 18 degree of freedom model as presented in \cite{Chaari2006} was implemented but proved to be difficult to solve. 
	
	
	%\bibliographystyle{plainnat}
	\bibliography{lib}
	%\bibliographystyle{agsm}
	
	
	
\end{document}
