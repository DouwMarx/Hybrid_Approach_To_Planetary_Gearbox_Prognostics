%%
%%  Department of Electrical, Electronic and Computer Engineering
%%  MEng Dissertation / PhD Thesis - Chapter 2
%%  Copyright (C) 2011-2016 University of Pretoria.
%%

\chapter{Literature Review/ Related work}
This literature review consists of two main parts. In the first part, concepts and methods that are important for the work documented in Chapter \ref{S:Practical} and \ref{S:Scope} are explained. The second part reviews prior studies relevant to a hybrid approach to prognostics for planetary gear systems.

\section{Explanation of concepts and techniques}

\subsection{Planetary gearboxes} \label{S:Literature}

Figure \ref{F:Planetary_Lei2014} shows an example of a planetary gear system. It consists of a ring gear and a sun gear that rotate around the same axis. The planet gears, that simultaneously mesh with the ring gear and the sun gear, rotate about their own axes whilst revolving around the centre of the sun gear. The planet gears, therefore, have unfixed centres whilst the ring and sun gears have fixed centres. A planet carrier rotating around the centre of the ring and sun gears maintain the spacing between planet gears. 
\begin{figure}[H]
	\centering
	\includegraphics[scale=0.7]{Chapter_2/Images_2/Planetary_Lei2014}
	\caption{A planetary gearbox \citep{Lei2014}}
	\label{F:Planetary_Lei2014}
\end{figure}



%\begin{figure}[H]
%    \centering
%        \includegraphics[scale=0.6]{Chapter_2/Images_2/Urbanek_2012}
%        \caption{Planetary gearbox in a wind turbine \citep{Urbanek2012}}
%        \label{F:Planetary_Lei2014}
%    \end{figure}

Various arrangements of planetary gearboxes are possible depending on which of the gearbox components are used as inputs and outputs and which of the components are kept stationary. I most cases, the ring gear is kept stationary and the input is either the sun gear or the planet carrier, depending on if a torque increase with speed decrease, or torque decrease with speed increase, is required. For the configuration of the planetary gearbox generally used in wind turbines, the ring gear is connected to a stationary housing with the input being the planet carrier and the output being the sun gear. The planet carrier drives that planets that mesh with the ring gear and transmit torque to the sun gear. This gear ratio leads to an increased rotational speed at the sun gear as compared to the planet carrier \cite{Zimroz2014}. 

The complex structure of planetary gearboxes invalidates the use of diagnostic methods that work for conventional fixed axis gearboxes \cite{Lei2014}. \cite{Lei2014} lists the following behaviours that are observed in planetary gearboxes.  

\begin{enumerate}
	\item  The multiple planet gears that simultaneously mesh with the sun and ring gears, lead to similar vibrations with different meshing phases. As a result, some of the vibrations cancel \citep{Blunt2006}.
	
	\item  The vibration transmission paths vary with time and lead to planet pass modulation \citep{Blunt2006}. Planet pass modulation leads to the suppression of frequencies generated by the planetary gear train that is not a multiple of the planet pass frequency \citep{McFadden1985}. The planet pass frequency is the frequency at which planet gears pass a stationary accelerometer mounted on the ring gear. 
	
	\item There are multiple transmission paths from the source of vibrations to the transducers. For instance, vibration due to a fault on the sun gear can be transmitted to the accelerometer in at least three ways: 1) Directly through the sun gear bearings to the casing, to the accelerometer. 2) Through the planet gears, to the ring gear, to the casing, to the accelerometer. 3) Through the planet gears, through the planet carrier, to the casing, to the accelerometer. These transmission paths also vary with time as the components in the gearbox rotate. A vibration signal originating from a faulty component that could be useful in estimating the condition of the machine is often masked by the attenuation and interference occurring in the vibration transmission paths. Furthermore, the torques and loads applied to the gearbox may lead to non-linear transmission paths \cite{Blunt2006}. These phenomena make the fault characteristics in a vibration signal obtained from a planetary gearbox difficult to interpret.
	%(Hines 2005 - This is not in the library) The whole transmission path bussiness
	
	\item Sidebands appear in the spectra of planetary gearboxes regardless of whether they are damaged or not. A sideband is a range of frequencies around the carrier frequency that is caused by modulation. These sidebands are asymmetric as opposed to the side-bands encountered in damaged fixed axis gearboxes, where the sidebands are typically symmetric about the gear mesh frequency (GMF). This is caused by multiple planet gears that produce similar vibrations, but with different meshing phases, that cause some of the gear meshing to neutralize. 
	
	\item The large transmission ratio of planetary gearboxes means that some components of the gearboxes operate at relatively low rotational speed. The low-frequency components are easily masked by noise, making it difficult to uncover faults in the vibration signal.
\end{enumerate}

Notice that a crack on one flank of a planet gear might manifest in a planet gear - ring gear mesh but not in a planet gear sun gear mesh and visa versa. A different side of the planet gear meshes with the ring gear teeth and sun gear teeth respectively. \cite{Lei2016}

\subsection{Uncertainty}

Uncertainties in sensor data can be classified into two categories: systematic departure due to bias and random variability due to noise. The former is caused by calibration error, sensor location, and device error, while the latter is caused by measurement environment \cite{Coppe2010}.



Notice that uncertainty in the operating conditions will probably also have to be captured with this.



\subsection{Particle filters}


\cite{Coppe2010}
It is generally accepted that uncertainty is the most challenging
aspect in prognosis [17,18]. Sources of uncertainty are from initial state estimation, current state estimation, failure threshold, sensor measurement, future load, future environment, and models. To address the uncertainty, various methods have been proposed, such as confidence intervals [19], relevancevectormachine [11], Gaussian process regression [11,12], and particle filters [15,20].



\cite{Orchard2008} states the particle filtering problem nicely and can be used to explain where all of the models fit into the big picture. Further they make use of all kinds of scary kernels and stuff 



In the prognostics of a planetary gearbox, the health state (hidden state) of the system needs to be estimated, given a series of measurements. With the health state known, the RUL can be estimated. Depending on the failure mode of interest, the health state of the system could, for instance, be the length of a root crack in a planet gear or the degree of pitting or spalling on the sun gear. Crack growth, pitting and spalling are all examples of gear failure modes. The overall system state can include both the health state and other system parameters. This means that unknown system parameters (such as Paris law coefficients in a crack propagation problem) can be updated as measurement become available. The measurements made in vibration-based condition monitoring are usually the acceleration signal measured by an accelerometer or some feature extracted from this acceleration signal. Examples of features that can be obtained from an acceleration signal that has been used in vibration-based condition monitoring are mean, variance, root mean square (RMS), skewness, kurtosis, shape factor, crest factor, and entropy \cite{Caesarendra2017}.

One way to perform this state estimation (updating the system state as measurements become available) is using a Kalman filter. However, the Kalman filter is based on the assumption of a linear system. Some degradation models (for instance the Paris crack propagation law) are non-linear, thereby invalidating the use of the Kalman filter. One way to address this non-linearity is to make use of the extended Kalman filter (EKF) that accounts for non-linearity in the system model through local linearisation \citep{Arulampalam2002}. Alternatively, a particle filter can be used. Particle filters implement the recursive state estimation problem using Monte Carlo simulations \citep{Arulampalam2002} and are not bound by the assumption of a linear system and Gaussian noise \citep{Gordon1993}. 

One popular variant of the particle filter that makes use of sampling importance resampling is the bootstrap particle filter. For a bootstrap filter, the density of the state vector is represented as a set of random samples that approximate the probability distribution of the state vector at some point in time. For the condition monitoring problem, the state vector would contain parameters relevant to system health state.  The random samples represent possible solutions to the state estimation problem and are updated and propagated by the algorithm as time progresses. 

In the recursive Bayesian recursive estimation problem, the state vector $x$ is assumed to evolve according to a system model $f_k$  as 
%Perhaps explain what Bayesian means in this context

\begin{equation}
x_{k+1}=f_{k}\left(x_{k}, w_{k}\right)
\label{eq:state}
\end{equation}

where $w_{k}$ is zero-mean white noise with a probability density function (PDF) that is assumed to be known.

As time progresses, measurements become available that are used to update the probability distribution of the state vector $x$. These measurements are related to the system state vector through the measurement model $h_k$, given as 

\begin{equation}
y_{k}=h_{k}\left(x_{k}, v_{k}\right)
\end{equation}

where $v_{k}$ is zero-mean white noise with a known PDF that represents the measurement noise. This measurement model can either be a data-driven model or an analytical expression.

The initial probability density function (prior) of the state vector is assumed to be known;

\begin{equation}
p\left(x_{1} | D_{0}\right) \equiv p\left(x_{1}\right)
\end{equation}

and the information that becomes available at a certain time-step $k$ is a set of measurements

\begin{equation}
D_{k}=\left\{y_{i} : i=1, \dots, k\right\}
\end{equation}

The purpose of Bayesian state estimation is to construct the PDF of the current state $x_k$ when given all available measurements. 

The first step in the Bayesian state estimation problem is called prediction. Given that the state PDF at time step $k-1$ is available, the system model can be used to predict the prior PDF of the state vector at time-step $k$ through the relation 

\begin{equation}
p\left(x_{k} | D_{k-1}\right)=\int p\left(x_{k} | x_{k-1}\right) p\left(x_{k-1} | D_{k-1}\right) d x_{k-1} .
\label{eq:first}
\end{equation}

The state evolution $p\left(x_{k} | x_{k-1}\right)$ is described by means of the the system equation, $f$ (Equation \ref{eq:state}).

Therefore,

\begin{equation}
p\left(x_{k} | x_{k-1}\right)=\int p\left(x_{k} | x_{k-1}, w_{k-1}\right) p\left(w_{k-1} | x_{k-1}\right) d w_{k-1}.
\end{equation}


Assuming that the system variance $w_k$ does not vary as the system state varies, $ p\left(w_{k-1} | x_{k-1}\right)=p\left(w_{k-1}\right)$ and therefore 

\begin{equation}
p\left(x_{k} | x_{k-1}\right)=\int \delta\left(x_{k}-f_{k-1}\left(x_{k-1}, w_{k-1}\right)\right)\times p\left(w_{k-1}\right) d w_{k-1}
\end{equation}


where $ \delta\left(.\right)$ is the Dirac delta function.


The second step of the Bayesian state estimation problem involves updating the prior distribution based on a measurement. As a measurement $y_k$ becomes available, the prior can be updated with Bayes rule as 

\begin{equation}
p\left(x_{k} | D_{k}\right)=\frac{p\left(y_{k} | x_{k}\right) p\left(x_{k} | D_{k-1}\right)}{p\left(y_{k} | D_{k-1}\right)},
\label{eq:second}
\end{equation}

where the denominator is given by 

\begin{equation}
p\left(y_{k} | D_{k-1}\right)=\int p\left(y_{k} | x_{k}\right) p\left(x_{k} | D_{k-1}\right) d x_{k}.
\end{equation}


The conditional distribution, $p\left(y_{k} | x_{k}\right)$ is defined using the measurement model with known measurement noise $v_k$ as

\begin{equation}
p\left(y_{k} | x_{k}\right)=\int \delta\left(y_{k}-h_{k}\left(x_{k}, v_{k}\right)\right) p\left(v_{k}\right) d v_{k}
\end{equation}


The bootstrap filter aims to approximate the distributions of equations \ref{eq:first} and \ref{eq:second} through Monte Carlo sampling. The posterior density function (The revised prior after a measurement is observed)  is represented by a series of random samples with associated weights \citep{Arulampalam2002}. 

The following steps are taken to implement the bootstrap particle filter.



%First iten was actually x_i and not x_1 initially
\begin{enumerate}
	\item Draw an initial sample of $N$ particles from the prior distribution $x_1(i) = \mathcal{N}(x_1,M_1)$, where $\mathcal{N}$ represents the multivariate normal distribution with mean $x$ and covariance $M$. Each particle represent a possible state of the system. 
	\item Make an observation and map it to the state vector space using the observation equation (measurement model) $h(x)$.
	\item Find the probabilities of the observation given each of the respective particle states, $P(y_k | x_k(i))$. This probability is dictated by the probability density $\mathcal{N}(y,v)$ where $v$ is the measurement noise and $y$, the measurement taken. To prevent numerical errors, log probabilities are used. The log multivariate normal distribution, $log(P(y_k|x_k(i)))=log(\frac{1}{\sqrt{2\pi \sigma ^2}}) - \frac{(x_k(i)-y)^2}{2\sigma^2}$ is used to gauge the probability of a particular measurement given the state of a certain particle.
	\item \label{initially} Find weights corresponding to the respective probabilities of each of the particles. The weights defined as $w_i = \frac{P_i}{\sum P_i}$ and gives an indication of how likely it is that a particular particle is at the true state of the system. When log probabilities are used, the numerator of $log(w_i) = \frac{log(P_i)}{log(\sum P_i)}$ is not simple to solve and a identity can be used to solve for the weights. The identity is given by $\log \sum_{i=0}^{N} a_{i} = \log a_{0}+\log \left(1+\sum_{i=1}^{N} e^{\left(\log a_{i}-\log a_{0}\right)}\right)$ where $a_0>a1>...>a_N$ (sorted in descending order). 
	\item Re-sample by re-selecting particles based on their computed weights. The re-sampling is done by sampling from a uniform distribution with a sample space between $0$ and $1$. The sample space is divided into sections corresponding to the size of the computed weights. Particles with high probabilities and therefore large weights are therefore more likely to be re-sampled (re-selected for the following time-step). 
	\item Compute the mean and standard deviation of the re-sampled particles as the current time-steps' prediction
	\item Repeat the procedure for each time-step. 
\end{enumerate} 

Take note that for the first iteration, the procedure should start at step \ref{initially}.

A problem encountered with sequential importance sampling methods like the bootstrap filter is that, after a few iterations, all particles apart from one particle will have a negligible weight. A large amount of computational resources is therefore used to update particles that do not contribute significantly to the approximation of the posterior distribution. This is known as the degeneracy problem \citep{Arulampalam2002}. This problem is addressed through resampling. When a significant amount of degeneracy is observed, some of the current particles are replaced with a new set of particles. 

\subsection{Gear dynamics and failure}
This section concerns gear failure modes and the dynamics of cracked gear systems. Considering that the detection of faults in gear systems is performed based on a change in the normal vibration characteristics, an understanding of gear faults and how they influence the vibration response is required.


%Linear elastic fracture mechanics assumes the material is isotropic and linear elastic.

\subsubsection{Gear failure modes}

In vibration-based condition monitoring, a faulty component would typically be identified through the process of diagnostics. After the type of fault is determined, a prognostics approach can be applied to estimate the RUL of the particular component. \cite{Lin2016} mentions the following main means of wind turbine gearbox failure in China.

\begin{itemize}
	\item Teeth surface pitting: The formation of fatigue cracks on the surface or just below the gear face. Pits are the results of these cracks breaking through the surface and causing material separation.
	\item Teeth bonding: Large stresses in sliding teeth lead to high transient temperatures and the bonding of teeth. This leads to scratches on the teeth flanks. This failure mode is more relevant to helical gears than spur gears. 
	\item Static indentation:  Indentations are formed due to large static forces during wind turbine downtime.
	\item Bearing damage: Several bearing failure modes such as inner race and outer race faults can occur. 
	\item Fracture: Fatigue gear tooth fracture can occur due to crack propagation under a cyclic load. Overload fracture can occur due to an excessive load on the gear tooth.
\end{itemize}

Ideally, an overarching prognostics strategy that accounts for all failure modes would be implemented. This would be very complex. A hybrid prognostics approach for wind turbine gearboxes in its simplest form would involve predicting the RUL of a component based on a single failure mode. The selection of the appropriate failure mode for a prognostics approach is therefore in order. A prognostics approach that estimates the tooth root crack length due to bending fatigue is reasonable. The majority of gear failure modes lead to a fatigue fracture problem near the end of life. Furthermore, fatigue cracks have the potential of resulting in catastrophic failures when a crack grows up to a point where the tooth breaks off completely. Cracks are initiated at stress concentration points where pitting and indentations are present and grow to ultimately lead to fatigue failure. Figure \ref{F:Ballarini} shows an example of tooth root crack growth due to bending fatigue. The crack is initiated at the location of maximum bending stress in the gear tooth.

\begin{figure}[H]
	\centering
	\includegraphics[width=0.5\textwidth]{Chapter_2/Images_2/Ballarini.png}
	\caption{Gear tooth root crack \citep{Ballarini1997}}
	\label{F:Ballarini}
\end{figure}


The propagation of a crack in isotropic and linear elastic materials is modelled by the Paris Law \citep{Paris1963}. The Paris law identifies the relationship between crack growth rate and stress state. The Paris law is given by 

\begin{equation}
\frac{d a}{d N}=C(\Delta K)^{m}
\end{equation}

where $\frac{d a}{d N}$ is the crack growth rate, $\Delta K$ is the stress intensity range and $C$ and $m$ are experimentally determined constants. The stress intensity $K$ is used to predict the stress state at a crack tip in a linear elastic material and is geometry, crack size and crack location dependent. Fatigue cracks generally have three distinct regions. The Paris law is applicable in the stable crack growth region where a log-log plot graph of $\frac{d a}{d N}$ versus $\Delta K$ is linear. If the constants that govern the Paris law is known, predictions can be made on the crack length given a certain loading condition. By using the Paris Law as a system model in the Bayesian state estimation problem, the constants in the Paris law can be updated as more measurements become available. By defining the end of life of a component as a certain crack length, the RUL of the component can be estimated. 




\subsubsection{Dynamics of cracked gear systems}
In vibration-based condition monitoring, the assumption is made that the measured vibration response characteristics of the machine is related to the severity of the fault under investigation. If the failure mode under investigation is crack gear tooth fatigue crack failure, the intention is, therefore, to measure the vibration response of the system to make an inference on the crack length. In a hybrid prognostics approach, physics-based models constitutes at least a part of this mapping between fault severity and the resulting vibration response. In the context of vibration-based, hybrid prognostics of planetary gear systems, it is important to understand the dynamics of cracked gear systems. 

\cite{Ma2015} states that the field of cracked gear systems consists of three subfields namely 

\begin{enumerate}
	\item Crack propagation prediction
	\item Time Varying mesh stiffness calculation
	\item Vibration response calculation
\end{enumerate}

These three fields are related as follows. A fatigue crack in a gear tooth will lead to a decrease in gear tooth stiffness \citep{Chaari2008}. This stiffness reduction is dependent on the length and the path followed by the crack making crack propagation path prediction relevant. The reduced gear tooth stiffness will influence the time-varying mesh stiffness (TVMS) and lead to a vibrations response different to that of and undamaged system \cite{Belsak2007,Ma2015}.


\textbf{Crack Propagation Prediction}

For most research on the crack propagation of spur gears, a $2D$ Finite Element Method (FEM) model and principles of linear elastic fracture mechanics is used to determine the crack propagation path. These 2D models are based on a plane stress or plane strain assumption and are more computationally efficient than their $3D$ counterparts. The assumption is therefore made that the crack length is uniform through the thickness of the gear. Other methods of determining the crack propagation path include Extended Finite Element Methods (XFEM), Boundary Element Methods (BEM), Analytical methods and the application of principles from elastic-plastic fracture mechanics. These aforementioned models are typically used to estimate the stress intensity factor (SIF) at the crack tip which is then used to determine the direction of crack propagation. Various methods used to calculate the SIF's include the displacement correlation method, the J-integral technique, the displacement correlation method, and the Paris law\cite{Ma2015}. The argument can be made that this subfield of the vibration of cracked gear systems is less relevant to the implementation of a hybrid prognostics approach for planetary gears if an assumption is made about the crack propagation path. Assuming the crack path to be a straight line or combination of several straight lines is a reasonable assumption \citep{Cheng2012,Zhao2013,Pandya2013}




%Here Cornell Fracture Groups FRANC2D software can be used. This software is distributed by Cornell Fracture group 


\textbf{Time varying mesh stiffness}

A gear tooth will have a certain stiffness depending on its material properties and geometry. As two gear gears mesh, they have a combined stiffness called the gear mesh stiffness. However, as the angle of the gears changes during the transmission of torque from the one gear to the next, the amount of gear tooth pairs in contact at an instant of time varies between a single pair and two pairs. As a result, the overall mesh stiffness varies with time as the number of gear teeth in mesh varies. This change in mesh stiffness with time results in a dynamic excitation. The time-varying mesh stiffness (TVMS) is dependent on several parameters including load, gear angular position, geometry, and gear contact ratio. Gear contact ratio is defined as the average number of teeth in contact during gear meshing with typical values ranging between $1.2$ and $1.6$. Figure \ref{F:Wan2014} shows an example of time varying mesh stiffness an how it is affected by the presence of a tooth root crack.


\begin{figure}[H]
	\centering
	\includegraphics[width=0.9\textwidth]{Chapter_2/Images_2/Wan2014TVMS.png}
	\caption{Time varying mesh stiffness with tooth root crack \citep{Wan2014}}
	\label{F:Wan2014}
\end{figure}


%You could measure gear mesh stiffness from the transmission error if you account for the torsional deflection of the gear shafts. 

There are various methods to determine the TVMS. Analytical methods are generally based on a potential energy principle and involve approximating the gear tooth as a variable cross-section cantilever beam. FEM can also be used to determine the TVMS. Various assumptions and simplifications are possible with the FEM method. This includes $2D$ plane strain or plane stress approximations, including contact elements between meshing gear teeth versus simply applying a force to a single gear tooth and dynamic simulations versus a quasi-static simulation. The third method of estimating the TVMS is a combined analytical and FEM method where the overall structural deformation is modeled using FEM and the local deformation is modeled using an analytical method. Finally, it is possible to estimate the time-varying mesh stiffness through experimental tests. Accurate measurement of the gear tooth deflections as required for this approach is however difficult to obtain. 


%$2x10^8 N/m$ is a typical gear mesh stiffness. 


\textbf{Dynamic models of gear systems}

A dynamic model is used to determine the expected vibration response for a given TVMS. Lumped mass models are mainly used to model the dynamics of cracked gear systems. However, finite element methods being are also used for this purpose when flexible shafts are modeled. 

The most commonly used lumped mass dynamical model is a two-dimensional model where each component has three degrees of freedom. A planetary gearbox with 4 planet gears will, therefore, have a total of 21 degrees of freedom where the planet gears, ring gear, sun gear, and planet carrier would each have two translational degrees of freedom and a single rotational degree of freedom. 


%There is also a 6DOF model by Bartelmus 

\section{Prior research}
This section gives a review of prior research in the condition monitoring of planetary gearboxes.

\cite{Lei2014} gives a review of the condition monitoring of planetary gearboxes and groups the different condition monitoring techniques for planetary gearboxes into three categories. The categories are modeling methods, signal processing methods, and intelligent diagnosis methods. 

Modeling methods aim to describe the relationship between the system output and the system parameters based on the knowledge of the physics of the problem. This category involves the simulation of various faults and vibration responses as well as investigating load sharing between planet gears. Some of the methods used to achieve this include phenomenological models, analytical models, multi-body dynamics models, and Fourier-series spectrum contributions. Research in this category aids in the understanding of the dynamics of planetary gearboxes. These models can be valuable as the physics-based component of a hybrid method. 

Signal processing is a popular method used in the condition monitoring of planetary gearboxes. This category involves processing the vibration signal using time-domain methods, frequency-domain methods or time-frequency-domain methods. Signal processing techniques aim to extract attributes from the signal that can indicate the damage level of the machine. Examples of time-domain methods include time-synchronous averaging techniques and the calculation of the signals' statistical features.
With frequency domain methods, the Fourier transform of a signal is calculated and used to diagnose faults based on the frequency components that are present in the signal. The signals generated by planetary gearboxes are non-stationary and therefore other preprocessing techniques are required before conducting frequency analysis.
Time-frequency methods simultaneously analyze a signal in the time domain and the frequency domain. This includes methods such as the Wigner-Ville distribution and wavelets. \cite{Lei2014} states that time-frequency methods are generally more effective in diagnosing planetary gearbox faults than time-domain or frequency-domain methods.

Intelligent diagnosis methods are largely data-driven, where the mapping between a vibration signal and the system health is based on a dataset serving as an example of the degradation of the machine. Examples of intelligent diagnosis methods include support vector machines, linear discriminant analysis, neuro-fuzzy inference, self-organizing neural networks, and K-nearest neighbour algorithms. Hybrid methods, that are the subject of this report are also categorized as intelligent diagnosis methods but comprise of a combination model-based (physics-based) and data-driven models. 


%Apparently, the planet separation technique is an important method of finding a valuable vibration signal.  

%Epicyclic time synchronous averaging. 

Articles that are closely related to the hybrid condition monitoring of a planetary gearbox are now presented. 

\cite{Li2005} \cite{Li200}

\cite{Cheng2012} presents a crack level estimation method for a sun gear tooth root crack in a 2K-H planetary gearbox. This method can be considered a hybrid method, incorporating both an analytical model for dynamic response calculation and a data-driven, grey relational analysis algorithm to estimate the damage level of the gearbox. An analytical model is used to estimate the gear tooth stiffness reduction as the crack (approximated as a straight line) grows. The lumped mass model used, assumes negligible central displacement of the planet gears, and the ring gear is ignored to further simplify the analysis. From the acceleration signal, $27$ commonly used features are computed and an optimal subset of these features is then weighted and used in the estimation of the crack length. \cite{Cheng2012} does, however, calibrate the model using a $0\%$ damage and a $100\%$ damage data point suggesting that run to failure data is a requirement for this method.

\cite{Orchard2007} presents an on-line particle filter framework for failure prognostics in a planetary gearbox carrier plate. A non-linear system model with unknown time-varying parameters is used to predict the evolution of a fault indicator and ultimately the system RUL within certain confidence bands. A two-step prognostics method is followed where the predictions are generated based on an a priori estimate, where after the RUL is estimated based on a set threshold. An axial crack in the planet carrier plate is grown in an experimental test. Unknown parameters in the Paris crack growth law was estimated using the particle filter. The particle filter measurement feature was based on the ratio between the fundamental harmonic and its side-bands. A non-linear measurement model is employed and periodically updated with the ground truth crack length. An additional investigation into the use of an EKF-approach proved that the particle filter was superior in accuracy and precision for the crack propagation problem. 


\cite{Patricks2007} introduces an integrated framework for the diagnosis and prognosis of a crack in the carrier plate of a helicopter gearbox. It involves the preprocessing of sensor data, the integration of model-based diagnostics and prognosis, the extraction of condition indicators and RUL prediction. A vibration model comprising of a finite element model of the carrier plate and a frequency response analysis is used to determine which vibration features are most effective in identifying the crack length.  The crack growth is characterized using the Paris-law. Real-time state PDF estimation is performed using a particle filter. A feature vector of observations is used to update the state PDF using a data-driven non-linear measurement model.


\cite{Zhao2013} develops an integrated prognostics method for gear tooth root crack length prediction. This method makes use of both physical models and real-time condition monitoring data. A 2D finite element model is used to analyse stresses at the gear tooth root. A 6-DOF gear dynamics model is used to calculate the dynamic load on the gear teeth given the mesh stiffness of a gear pair subjected to a certain crack length. The gear mesh stiffness used in this model is calculated using a potential energy approach and a curved crack path (approximated by straight lines). The dynamic load from the dynamics model is used in the FEM to obtain the SIF and therefore gain insight into the crack propagation. The basic Paris law is used as damage propagation model and a Bayesian method is used to update the state distributions and predict the RUL. Three sources of uncertainty are considered in the state estimation problem, namely material uncertainty, model uncertainty, and measurement uncertainty. \cite{Zhao2013} argues that a crack in a gear tooth would reduce the meshing stiffness and as a result, the dynamic load on the gear tooth will be influenced. The maximum dynamic load for a certain crack length is then used to determine the amount of crack growth. 


\cite{Zhao2018a} presents an integrated prognostics method for the surface wear gear failure mode in a planetary gearbox. The parameters in the Achard wear model is updated through gear mass loss inspection data using a Bayesian update process. Achard's wear model makes use of sliding distance and contact pressure which is calculated based on the gear mesh geometry and the contact process. Both a Gaussian and an uninformative prior is considered for the investigation. The uninformative prior achieved similar parameter results as compared to the Gaussian prior, illustrating that this can be an effective prognostics approach even without exact knowledge of the initial level of damage. 

%\textbf{Cheng 2015}


% Please add the following required packages to your document preamble:
% \usepackage{booktabs}
% \usepackage{graphicx}

\cite{Liao2016} proposes a hybrid method for Lithium-Ion battery condition monitoring that incorporates a physics-based model and two data-driven models. The physics-based method (Particle filter) describes the system degradation model. The first data-driven model is used to create the measurement model. The second data-driven model is used to predict future measurements for use in the physics-based method. This paper is discussed in more detail in Section \ref{S:Liao battery}

\cite{He2012} presents an integrated approach for spiral bevel gear health prognostics using particle filters. A data driven model (ARMIA) is used to model the measurement function. Another data driven model (double exponential smoothing) is used to model the state transition function. A Cholesky decomposition based whitening transform is used to convert the measured data into a one dimensional heath indicator. Inductance oil debris sensors and accelerometer measurements are taken. The time synchronous average was computed and may condition indicators were computed using the time synchronous average. Correlation coefficients of the condition indicators with time was calculated to determine which condition indicators show a good trending correlation. 

\section{After MSS review}


\subsection{Yu 2017}
Yu 2017 Compares three commonly-used methods used to esitmate the gear mesh stiffness of gears. The three methods are the finite element method, an analytical method based on a potential energy approach and a method based on the ISO standard. 

Finite element are the most popular method and is used as a bechmark to verify the correctness of other methods

Analytical methods is less computationally expensive than finite element methods and provide satisfying results. A commponly used analytical method is based on the potential energy approach where the gear tooth is approximated as a non-uniform cantilever beam. 

Finally a approximation using the ISO standard is also possible. It has the advantage of being simple and giving a simple order of magnitude estimation of the GMS. 

Finds that the stiffness in single tooth contact region is very similar between methods  but there are significant discrepancies in the double contact region. 

The traditional Analytical method an the ISO approximation method overestimates the mesh stiffness of multiple tooth pair contact

The Improved analytical method gives results that are relatively consistent with the FE methods. 

Differnece between calculation results of 2D and 3D Fem are within 10% proving that the 2D simplification makes sense

Recommends makeing use fo the ISO standard method only when the order of magnitude of the GMS is of interest in for instance the design stage of a planetary gearbox. 


\subsection{Shao 2013}





\subsection{Li 2005}
\cite{Li2005a} presents a model based remaining useful life estimation strategy for a gear with a fatigue crack. The stress intensity factors and mesh stiffness for a series of crack lengths are computed with FEM. The meshing stiffness is assumed periodic and is approximated with a truncated Fourier series. The computed stiffness is used in a dynamic model. A least squares optimisation problem is solved to find the optimal mesh stiffness for a given torsional vibration measurement based on transmission error readings. \cite{Li2005a} argues that as the root crack changes the tooth stiffness, the meshing dynamics will change and consequently the dynamic load on the gear tooth will change. This calculated dynamic load is then used to compute the SIF's in a FEM in order to ultimately compute the RUL.




\subsection{Coppe 2010} %2017?
Coppe 2010 Does structural health monitoring based on sensor data for fatigue-induced damage. 

Bayesian inference is used to reduce the uncertainty of damage growth parameters and is applied to the damage growth in fuselage panels

Proves that a large anount of data can be used to characterize the damage growth behavious of a specific structure.

Investigates a through thickness crack in a aircraft fuselage panel that grows under cycles of pressurization.

\subsection{He 2019}
Mentions that sun gears in planetary gearboxes are designed to float in the radial direction to ensure that there are uniform load distribution among different planet gears. This floating sun gear can easily be wrongly diagnosed as a distributed defect. The gearbox in this investigation does have a fixed sun gear but floating planet gear carrier assembly. This misdiagnosis could present a problem. Feature indicators are developed that can be used to avoid the misdiagnosis of the floating sun gear.


%% End of File.
