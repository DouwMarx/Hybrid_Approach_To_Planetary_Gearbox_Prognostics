%%
%%  Department of Electrical, Electronic and Computer Engineering
%%  MEng Dissertation / PhD Thesis - Chapter 5
%%  Copyright (C) 2011-2016 University of Pretoria.
%%

\chapter{LMM}
%\chapter{Discussion}

\section{Lumped mass modelling of planetary gearbox}
Post MSS lumped mass literature review


General introduction:
Vibration of planetary gearbox is more complicated than fixed axis. Vibrations generated are similar but have different phases. 

The transmission path changes due to the carrier motion.

Dynamic simulation is used rather than mathematical models seeing that the physical parameters of the gearbox like the gear mesh stiffness can be incorporated in the model \cite{Liang2015}

--------------------------------------------------------------

\cite{Ma2015} provides a review on cracked gear systems

\cite{Liang2015} develops a dynamic model to simulation vibration signals and makes use of a Hamming function fo represent the transmission path. Uses a model similar to cite (lin and parker 1999). Differences include describing planet deflections in horizontal and vertical coordinates, considering gyroscopic and centrifugal force and incorporating accurate model parameters.

\cite{Chen 2015} makes use of dynaic model as used in chaari 2006  and 4 chen 2013 articles.

\cite{Chen2013} makes use of matrices as in Chaari 2006 and Parker 2000

In this model we will not be considering flexible ring gear as in \cite{Chen2015}


A commonly used lumped mass model in literature is that of \cite{Kahraman1994} which has been used with some adaption by  \cite{Lin1999}, \cite{Chaari2006}, \cite{Chen2013},\cite{Chen2015}. The main focus of this investigation will be the specific model used by \cite{Lin1999}. In this model, the sun, ring, carrier and planets are all treated as rigid bodies with bearings and gear meshes modelled as linear springs. Each component of the gear box has two translational degrees of freedom in plane and one rotational degree of freedom around an axis orthogonal to the translational plane. No damping is explicitly modelled in the system. Rather, a proportional damping is considered as in \cite{Chaari2006}.

Gyroscopic effects are this fraction of the typical stiffness matrix an will be neglected or not. Or gyroscopic stiffness is included although it does not really do anything. 

All translations of components are defined relative to a rotating reference frame fixed to the carrier as such, the vibration response obtained from the lumped mass model is not directly comparable to the response measured by an accelerometer mounted to the gearbox housing. The transmission path from the gearbox components in the rotating reference frame to the measured response of the accelerometer therefore needs to be modelled.

\cite{Liang2015} considers the resultant vibration signal to be a weighted summation of each planet gear, arguing that, as the planet gear meshes with the sun gear and ring gear simultaneously, the planet gear vibration contains information about both the sun-planet and ring-planet meshes. The effect of the transmission path is modelled by a modified Hamming function with $\alpha$ parameter that controls the bandwidth of the Hamming function. The equation is given by 3.1 in the paper.

\cite{Parra2017} on the other hand states that they consider ring and planet vibrations and says liang does not consider ring vibrations as well. 
--------------------------------------------------------------


In this section, a $21$ DOF lumped mass model of a planetary gearbox \citep{Chaari2006} is programmed. Each component of the gearbox has two translational degrees of freedom and a rotational degree of freedom.

Figure \ref{F:model} shows a schematic of the lumped mass model. 

\begin{figure}[H]
	\centering
	\includegraphics[width=0.6\textwidth]{Chapter_3/Images_3/chaari2006}
	\caption{Planetary gearbox model \citep{Chaari2006}}
	\label{F:model}
\end{figure}

The equations of motion of the lumped mass system is given by 

\begin{equation}1
M \ddot{x}+\Omega_{c} G \dot{x}+\left[K_{b}+K_{e}(t)-\Omega_{c}^{2} K_{\Omega}\right] x=T+F(t)
\end{equation}

where 
\begin{itemize}
	\item $x$ is the vector of degrees of freedom
	\item $M$ is the mass matrix
	\item $\Omega_{c}$ is the rotational speed of the planet carrier
	\item $G$ is the gyroscopic matrix 
	\item $K_{b}$ is the bearing stiffness matrix
	\item $K_{e}(t)$ is the time varying mesh stiffness matrix
	\item $T$ is the vector of external torques
	\item $F(t)$ is the excitation force vector due to factors such as transmission error        
\end{itemize}

The expressions for each of the respective matrix indices are documented in \cite{Chaari2006}.

%For the chaari example we are running at $857 RPM$, The GMF is calculated as $300Hz$. the time varying mesh stiffness is approximated as a square wave. The TVMS can however be fed to the model as any arbitrary profile. A gear tooth crack can therefore artificially added to the model. An example of the time varying mesh stiffness depending on the contact ratio is shown in Figure \ref{F:TVMS}

The TVMS for the sun planet and planet ring interaction for a healthy gear system is approximated as a square wave. An example of the TVMS for the gearbox modelled in \cite{Chaari2006} running at $857RPM$ (GMF = $300Hz$) is shown in Figure \ref{F:TVMS}.


\begin{figure}[H]
	\centering
	\includegraphics[width=0.7\textwidth]{Chapter_3/Images_3/TVMS}
	\caption{TVMS approximation}
	\label{F:TVMS}
\end{figure}


The overall stiffness matrix is given by

\begin{equation}
K = K_{b}+K_{e}(t)-\Omega_{c}^{2} K_{\Omega}.
\end{equation}

The overall mesh stiffness matrix is plotted in Figure \ref{F:Stiffness}. Since the mesh stiffens is time-varying the overall stiffness matrix will also be time-varying and there will be two cases of the stiffness matrix for a healthy gearbox under the square wave mesh stiffness assumption. 

\begin{figure}[H]
	\centering
	\begin{subfigure}{0.49\textwidth}
		\includegraphics[width=1.1\textwidth]{Chapter_3/Images_3/Total_stiff1.pdf}
		\caption{time = $0s$ }
	\end{subfigure}
	~
	\begin{subfigure}{0.49\textwidth}
		\includegraphics[width=1.1\textwidth]{Chapter_3/Images_3/Total_stiff2.pdf}
		\caption{time = $0.0025s$ }
	\end{subfigure}
	
	\caption{Stiffness matrix for model used by \cite{Chaari2006}}
	\label{F:Stiffness}
\end{figure}

Notice that there are non-negative terms on some of the off-diagonal elements of the stiffness matrix (from \cite{Chaari2006} equation 5). This results in a non-positive definite stiffness matrix. As a result, obtaining the solution of the lumped mass model through numerical integration, even with the addition of proportional damping, is very difficult.  

As a simple initial validation of the model, the eigenfrequencies of the model based on the average mesh stiffness is calculated by the eigenvalue problem. However, \cite{Chaari2006} does not explicitly state the gear mesh ratio which is required to determine the average mesh stiffness. The lumped mass model therefore still needs to be verified using \cite{Chaari2006} or \cite{Lin1999}.


%%%%%%%%%%%%%%%%%%%%%%%%%%%%
\cite{Xiang2017} Dynamic analysis of a planetary gear system with multiple nonlinear parameters.

\cite{Xiang2017} models a planetary geartrain and models TVMS gear, gear error and backlash nonlinearities. The dynamic responses are solved using 4-5 order variable step Runge-Kutta numerical integration. Good overview of planetary gearbox modelling history is given. Considers a purely rotational model of a multistage gear system. Constitutes a planetary gear stage and two parallel gear stages.

First 1500 revolutions are discarded to reach steady state conditions.

Mentions that the gear meshing frequency is 

\begin{equation}
	\omega_{\mathrm{spi}}=\omega_{\mathrm{c}} Z_{\mathrm{r}}
\end{equation}



%% End of File.
