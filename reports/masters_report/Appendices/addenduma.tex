%%
%%  Department of Electrical, Electronic and Computer Engineering
%%  MEng Dissertation / PhD Thesis - Addendum A
%%  Copyright (C) 2011-2016 University of Pretoria.
%%

\chapter{Gearbox test bench design considerations } \label{A:testbench}

This section discusses design considerations for the planetary gearbox test bench. Motivations for not choosing the listed alternative designs are provided.

\textbf{Using a single gearbox}

Ideally, a single gearbox will be used without the need for an un-monitored, step down gearbox to increase the input torque. This will result in a simple setup and cleaner accelerometer measurements. Given that the gearbox is to be used in the speed up configuration, the motor starting torque is estimated at $40Nm$ and the hydraulic load has an operating range up to $6000RPM$, the singe gearbox layout seems like a good one. Furthermore, this configuration would lead to a minimal amount of alignment time after the gearbox is opened up to make modifications. A test, however, proved that even without any load, the static torque required to turn the gearbox in its speed up configuration is too high for the DC motor. 


\textbf{Using an electrical load}

By using an electric load instead of a hydraulic load, the general test setup would have been cleaner and less cluttered. Furthermore, research in the field of motor and stator current monitoring \citep{Nie2013} suggest that equipping the planetary gearbox test bench with an electric load could allow for future research in this field. An electric load was not used since a motor drive or resistor bank for the electric load was not available and would have to be bought. Furthermore, the electric motors that were available to be used as load typically have an upper-speed limit of $1500RPM$. This means that, for a single gearbox in the speed up configuration, the motor would have to operate between $0\%$ and $9\%$ of its maximum rotational speed of $3000RPM$. Although this operating range would ensure that the DC motor is capable of producing large torques, there is a risk of the motor running very inefficiently or burning out. The other, more feasible alternative would be using two gearboxes in speed down, speed up configuration. However, the $1500RPM$ load RPM ceiling will still mean that the DC motor will run at only half its maximum speed of $3000RPM$.


\textbf{Measure the torque load by making use of a torque transducer}

An accurate estimate of the torque reading would allow for the implementation of a feedback controller that controls the solenoid valve setting and ultimately, the hydraulic load. Wind turbine loading cases could then be accurately simulated in this configuration. The cost of buying a torque transducer to take accurate torque measurements will not be justifiable, considering that the motor current can be measured from the variable frequency drive, giving (although less precise) an indication of the torque produced by the motor. 

%\textbf{Test bench assumtions}
%Given that a wind turbine typically revolves at 14RPM
%Dealing with a much stiffer setup than encountered in wind turbines


\chapter{Simplifications made to implement the hybrid model} \label{A:simplifications}

The following simplifications and alterations were made to the procedure by \cite{Liao2016} to focus the investigation on the hybrid method, rather than the condition monitoring of Lithium-Ion batteries.
\begin{enumerate}
	\item The internal resistance measurements were used directly and not estimated using a lumped parameter model as in \cite{Liao2016}
	\item \cite{Liao2016} built build data-driven models based on several degradation datasets where a single additional dataset was used to create the data-driven models in this investigation. 
	\item \cite{Liao2016} made use of a similarity-based approach for measurement prediction while a particle filter was used here
	\item The dataset did not provide resistance and capacitance values at every cycle. A new dataset with resistance and capacitance readings at every cycle was created by interpolating. 
\end{enumerate}






\chapter{Hybrid model variants and their applicability to planetary gearbox prognostics} \label{A:Hybrid}

Table \ref{t:Hybrid_Compare} list a few ways in which a hybrid prognostics approach could be applied to planetary gearboxes
\begin{table}[]
	\centering
	\caption{Hybrid modelling strategies}
	\label{t:Hybrid_Compare}
	%\resizebox{\textwidth}{!}{%
	\begin{tabular}{|p{4cm}|p{4cm}|p{7cm}|}
		\hline
		\textbf{Hybrid method }& \textbf{Explanation} & \textbf{Applicability to planetary gearbox RUL prediction} \\ \hline
		\textbf{Data driven method to infer measurement model and model-based method to predict RUL}& The measurement model can be very non-linear and difficult to model analytically & This method can be valuable if the data-driven measurement model can be trained on healthy data and still apply to unhealthy gearbox operation. An example would be learning characteristics of the vibration transmission path\\\hline
		\textbf{Data-driven method to replace a system model in a model-based RUL prediction method} & A data-driven system model can reduce the effort of modelling complex system physics & Obtaining enough data for the model to be valid at all possible operating conditions is impractical. Furthermore, the existing domain knowledge about planetary gearbox failure modes will remain unused\\\hline 
		\textbf{Use a data-driven method to predict future measurements that are then used in a model-based method    }        & The method generates measurements for model-based methods (particle filter) when long term predictions are made. &The accuracy to which future measurements can be predicted can have a strong influence on the predicted RUL. The planetary gearbox vibrations will have to be predicted accurately without the availability of run to failure data\\ \hline
		\textbf{ Combine a data-driven model and a physics model by averaging their results}& The results of a data-driven approach and physics-based approach are combined to exploit the advantages of both  & The averaging mechanism is not simple to design. Furthermore, it will be very difficult to use a purely physics-based approach for a complex system such as a planetary gearbox\\ \bottomrule
	\end{tabular}%
	
\end{table}

%% End of File.A:Gear_Geom


\chapter{Particulars of gear geometry} \label{A:Gear_Geom}

The results generated by the gear geometry generator \cite{KHK2015} is shown for each gear in the planetary system in Figure \ref{F:Planet_Ring_Geom} and \ref{F:Planet_Sun} respectively.

\begin{figure}[H]
	\centering
	\includegraphics[scale=0.9]{Appendices/Images_A/iter2ringplanetb.png}
	\caption{Planet-Ring geometry specification}
	\label{F:Planet_Ring_Geom}
\end{figure}

\begin{figure}[H]
	\centering
	\includegraphics[scale=0.9]{Appendices/Images_A/iter2sunplanetb.png}
	\caption{Planet-sun geometry specification}
	\label{F:Planet_Ring_Geom}
\end{figure}
 		