\documentclass[]{article}
\usepackage{graphicx}
\usepackage{float}
%\usepackage{natbib}
\usepackage{booktabs}
%opening
\title{Lumped mass model selection for use in a hybrid approach to prognostics in a planetary gearbox}
\author{}

\usepackage[utf8]{inputenc}
\usepackage[english]{babel}
\usepackage{listings}

\usepackage{amsmath}
\usepackage{amsfonts}
%\usepackage{aligned}

%Import the natbib package and sets a bibliography  and citation styles
\usepackage{natbib}
\bibliographystyle{abbrvnat}

\setlength{\parskip}{1em}


\begin{document}
	
\section{Bayesian state estimation problem}

The Bayesian state estimation problem consists of two parts. During model based prediction, an a-priori estimate of a future state is obtained by projecting the current state though a system model that attempts to model the physical system. As a measurement containing information about the actual system state becomes available, the current belief of the system state, as expressed by the prediction step, can then be updated with the measurement information in the update step \citep{Fang2018}. 

Consider the nonlinear discrete-time system:
\begin{equation}
\left\{\begin{aligned}
x_{k+1} &=f\left(x_{k}\right)+w_{k} \\
y_{k} &=h\left(x_{k}\right)+v_{k}
\end{aligned}\right.
\end{equation}

\begin{itemize}
	\item $x_{k} \in \mathbb{R}^{n_{x}}$ is the system state: A vector of variables that should fully describe the status or behaviour of the system. $n_{x}$ is a positive integer.
	\item $y_{k} \in \mathbb{R}^{n_{y}}$ is the system output that contains some information about the true system state. $n_{x}$ is a positive integer.
	\item $f: \mathbb{R}^{n_{x}} \rightarrow \mathbb{R}^{n_{x}}$ is a nonlinear mapping that represents the process dynamics or how the state vector changes with time. 
	\item $h: \mathbb{R}^{n_{x}} \rightarrow \mathbb{R}^{n_{y}}$ is a nonlinear mapping that represents the measurement model. The full system state is often not directly measurable and the measurement model is therefore used to infer the system state $x_{k}$ from the system output $y_{k}$.
	\item $w_{k}$ is the process noise, $v_{k}$ is the measurement noise. They are used to express uncertainty in the model output and the measurements made. They are mutually independent zero-mean white Gaussian sequences with covariances $Q_{k}$ and $R_{k}$ respectively.
\end{itemize}

As there is uncertainty in the system state, it is modelled as a random vector. Similarly, as a measurement becomes available at time k it can be viewed as a sample drawn from the distribution of the random vector $y_k$.



The prediction step in the Bayesian state estimation problem is given as 
\begin{equation}
p\left(x_{k} | \mathbb{Y}_{k-1}\right)=\int p\left(x_{k} | x_{k-1}\right) p\left(x_{k-1} | \mathbb{Y}_{k-1}\right) \mathrm{d} x_{k-1}
\end{equation}

where 

\begin{itemize}
	\item $p\left(x_{k} | \mathbb{Y}_{k-1}\right)$ is the a priori estimate of the state $x_k$ before the information of a measurement has been incorporated. 
	\item $p\left(x_{k} | x_{k-1}\right)$ is the probability of a new state, knowing the previous state. The nonlinear process dynamics mapping $f$ would ideally help to accurately describe this distribution.
	\item $p\left(x_{k-1} | \mathbb{Y}_{k-1}\right)$ is the previous posterior estimate of the state vector with $\mathbb{Y}_{k}$ a set of previous measurements $\mathbb{Y}_{k}=\left\{y_{1}, y_{2}, \cdots, y_{k}\right\}$ . This distribution has therefore been updated with the some information obtained from measurement. 
\end{itemize}

The update step is given by

\begin{equation}
p\left(x_{k} | \mathbb{Y}_{k}\right)=\frac{p\left(y_{k} | x_{k}\right) p\left(x_{k} | \mathbb{Y}_{k-1}\right)}{p\left(y_{k} | \mathbb{Y}_{k-1}\right)}
\end{equation}

where 

\begin{itemize}
	\item  $p\left(x_{k} | \mathbb{Y}_{k}\right)$ is the a-posteriori estimate of the state vector incorporating the latest measurement information. 
	\item $p\left(y_{k} | x_{k}\right)$ is the probability of seeing the current measurement considering that the current state, from the prediction step, is expected to be $x_k$. The measurement model $h$ should give an indication of this probability. 
	\item $p\left(x_{k} | \mathbb{Y}_{k-1}\right)$ is the a priori estimate resulting from the prediction step.
	\item $p\left(y_{k} | Y_{k-1}\right)$ is the probability of the new measurement given all of the previous measurements. 
\end{itemize}


Although the aim of the diagnostics algorithm is to ultimately estimate the current health state (crack length) given the acquired measurements, there are uncertainties with regards to the model parameters as well. The incorporation of these model parameters into the state vector together with the health state is known as simultaneous state and parameter estimation problem\cite{Fang2018}

For the crack prediction problem there are numerous uncertainties in the process dynamics model $f$ and measurement model $h$.

\begin{itemize}
	\item Crack Propagation Fem
	\begin{itemize}
		\item Paris Law Parameters $C$ and $m$.
		\item Applied load magnitude position and angle. 
		\item Material properties $E$ and $\nu$
		\item Crack initiator position, angle and length.
		\item Variations due to FEM mesh sensitivity. 
	\end{itemize}
	\item Time varying mesh stiffness FEM
	\begin{itemize}
		\item Cracked mesh: The crack lengths tested are not continuous
		\item Applied moment as measured through required motor current.
		\item Material properties $E$ and $\nu$
		\item Friction coefficient
		\item Gearbox geometry: Centre distance, involute profile, manufacturing errors
		\item Overall uncertainty in modelling simplifications
		\item Variations due to FEM mesh sensitivity. 
	\end{itemize}
	\item Lumped mass model 
	\begin{itemize}
		\item Bearing stiffness
		\item Uncertainties from model simplifications
		\item Effective mass of lumped masses
		\item Numerical accuracy of solution to LMM differential equation
	\end{itemize}
	\item Signal processing and Feature extraction
	\begin{itemize}
		\item Uncertainty in time synchronous averaging techniques
	\end{itemize}
	\item Data driven mapping between physics based model and reality
	\begin{itemize}
		\item Uncertainty that data driven model will no longer be applicable when the system state changes ie. the system is being in a less healthy state than when the data driven mapping was trained and thereby having a different mapping between the actual measurements and the physics based model.
	\end{itemize}
\end{itemize}


For each of the uncertainties in the model, one could define a Gaussian distribution with mean the expected value of a parameter and the standard deviation expressing the degree of uncertainty in the actual value of the parameter. 

By drawing a a sample from each of the parameter distributions and feeding these parameters into the nonlinear process dynamics model $f$, one can obtain a single sample of the distribution $p\left(x_{k} | x_{k-1}\right)$. By drawing a very large number of samples from the parameter distributions, and approximation of $p\left(x_{k} | x_{k-1}\right)$ can thereby be obtained.

Similarly, by feeding a large amount of samples from the parameter distributions that affect the measurement model $h$, though the model, an estimate of the probability distribution of $p\left(y_{k} | x_{k}\right)$ can be obtained.

The computational cost of the above approach would however be very large seeing that the fem simulations would each have to be run several times.

\textbf{1) Could this sampling approach make a sense? Are there ways to know how many samples your would require to approximate the distribution well?}

Problems associated with the computational cost  can be addressed by considering only the variables with a high degree of uncertainty or those which would most greatly affect the effectiveness of the state estimation. This means variables with a high certainty are simply fixed as a constant non-random variable. Furthermore a Gaussian assumption can be made and some variant of the Kalman filter can be applied (KF, EKF, UKF, EnKF)

A typical crack growth estimation problem would have a state vector
\begin{equation}
x =	\left[\begin{array}{l}
a
\end{array}\right]
\end{equation}

Where $a$ is the tooth root crack length. Important Paris Law parameters could also be incorporated in the state vector leading to a simultaneous state and parameter estimation problem.

\begin{equation}
x = \left[\begin{array}{l}
a \\
m\\
C
\end{array}\right]
\end{equation}

\textbf{2) Why is the time derivative of the health state ($\dot{a}$) not included in the state vector as in many dynamical systems?} 

The measurement vector consists of several features.

\begin{equation}
y =	\left[\begin{array}{l}
F1 \\
F2 \\
F3
\end{array}\right]
\end{equation}

Where $F1$ is some vibration feature such as RMS or kurtosis.


The state transition function would then be derived from the Paris Law (\cite{Zhao2013} for example).

\begin{equation}
\begin{array}{l}
a((i+1) \Delta N)=a(i \Delta N) \\
+(\Delta N) C[\Delta K(a(i \Delta N))]^{m} \varepsilon, \quad i=0,1,2, \ldots, \lambda-1
\end{array}
\end{equation}

where 

\begin{itemize}
	\item $\Delta N$ is the time increment in number of cycles
	\item $\Delta K$ is the stress intensity factor as calculated by a FEM simulation
	\item $\varepsilon$ is Gaussian noise to compensate for modelling errors. 
	\item $C$ and $m$ are Paris law parameters that could also be included in the state vector. 
\end{itemize}



\textbf{3) Why is the Paris law used as state transition model $h$ rather than a data driven model of the FEM crack growth result. The FEM simulation could be more representative than the Paris law and is computed already to obtain the appropriate and mesh for TVMS calculation and the SIFs for crack growth.}


\textbf{4) Instead of gradually gaining certainty about the Paris Law parameters, could you rather gradually gain certainty about which of the previously ran FEM crack simulations (with a design space of various Paris law parameters) are the most relevant given the measurements?. This could also be formulated as an optimisation problem: Given the measurements to date, which of the crack growth FEM simulations and LMM parameters do I expect to give to most accurate RUL prediction?}

\textbf{5) Seeing that crack growth is a rather slow process, could it make sense to directly incorporate the FEM inside the state transition model $f$ and evaluate the FEM at each time step? Could the model uncertainty be dominated by the uncertainty of a single parameter rendering a certain sophisticated model section meaningless?}

\textbf{6) Why not track the RUL as health state directly rather than the crack length? In fact, $x_{k}$ could either be the crack length, mesh stiffness or RUL? RUL would be a convenient health state as this is the variable we are ultimately looking for. Furthermore, the Ground truth RUL could possibly be more easily measureable (using the Hydropuls machine cycle counter) while measuring crack length with the microscope leads to ground truth measurement error.}


The measurement function $h$ would consist of a TVMS FEM simulation surrogate model feeding into a lumped mass model. Finally features would be computed from the computed LMM response. The measurement noise associated with $h$ would be assumed Gaussian. 

\textbf{7) Would it make sense to apply another Bayesian updating procedure on the measurement model parameters using healthy data to estimate the measurement model parameters? (Rather than using the healthy data for a data driven mapping between the physics based model and reality) This would also give measurement uncertainty that can be used in the state estimation problem.}

	
	
	
\end{document}
