\documentclass{beamer}

\usepackage[utf8]{inputenc}
\usepackage{amsmath}


\newcommand{\norm}[1]{\lVert#1\rVert}


%Information to be included in the title page:
\title{Sample title}
\author{Anonymous}
\institute{Overleaf}
\date{2014}



\begin{document}

\begin{frame}
\frametitle{Agenda}
\begin{itemize}
	\item Beplande Eksperimentele Toetse
	\item TSA, Order tracking
	\item Cost function/Optimisering
	\item LMM, Stiff-Solvers
	\item Admin
\end{itemize}
\end{frame}

\begin{frame}
\frametitle{Beplande Eksperimentele toetse}
\begin{itemize}
	\item Make cut in gear tooth at maximum bending stress location. (Waarskynlik nie meer EDM nie gegewe omstandighede)
	\item Grow crack in Hydropuls machine under high load until crack visibly grows $\pm 0.5mm$ and record cycles required.
	\begin{itemize}
		\item Maak dit sin om toetse te doen deur te werk met kraakgroei in lineêre inkremente? 
		\item Gedagte om eerder in aantal siklusse te werk ter wille van moontlike afstand meting onakkuraatheid?
	\end{itemize}
	\item Insert Cracked gear in gearbox and run the setup for 5min to reach opperating temperature.
	\end{itemize}
\end{frame}

\begin{frame}
\frametitle{Beplande Eksperimentele toetse}
\begin{itemize}
	\item Stop running. Align the cracked tooth with one a known ring gear tooth to ensure that the applicable TSA averaging window can be deduced from the 1xper rev shaft encoder signal.
	\begin{itemize}
		\item Maak dit sin om hierdie sit te doen? Ek is bang dat dit moeilik is om vir klein kraaklengtes te bepaal watter tand die fout het met net die vibrasie syn en wil probeer om die revolusies te tel dat ek weet presies met watter rat tand ons te doen het. 
		\item Dit gaan moeilik wees om die son rat elke keer presies dieselfde op te lyn en ek dink nie mens sal met sekerheid kan se dat presies dieselfde rat tande vir elke eksperiment met mekaar paar nie. 
	\end{itemize}
	
	\item Start actual test and perform all tests at different loads in a single run. Change loads after fixed time period and allow for some time to reach steady state in between tests.
		\begin{itemize}
		\item Die gedagte hier is on te verseker dat die posisie van die planeet rattande bekend is vir alle toetse by verskillende las.
	\end{itemize}
    \item Repeat until failure
\end{itemize}
\end{frame}


\begin{frame}
\frametitle{TSA}
\begin{itemize}
	\item  TSA vir die planetêre ratkas behels om 'n venster van die vibrasie te onttrek elke keer as die planeet verby die versnellingsmeter beweeg. Die gemiddeld van die vibrasie word dan bereken volgens die mesh sequence.
	\item  Order tracking is gedoen om TSA se vensters meer vergelykbaar te maak. 
	\item Die order tracking het die TSA verbeter in die sin dat die vensters se inhoud nie meer gemeng is in sekere dele nie. Die gedagte was om teen konstante spoed te toets maar daar is tog klein fluktuasies. Onder watter omstandighede kan order tracking meer skade as goed doen? 
\end{itemize}



\end{frame}

\begin{frame}
\frametitle{Cost function}
\begin{itemize}
	\item y-rigting vibrasie vir "fout-styfheid" word vergelyk met een venster van TSA syn. 
	\item Het nou 'n goeie raamwerk vir optimeringsprobleem.
	\item Maak dit sin om die z telling te gebruik om die eksperimentele en analitiese wêrelde te vergelyk (syn - gemiddeld)/standaard afwyking? 
	\item Het nog nie konkrete bewyse dat die LMM gestadigde toestande bereik nie maar ek vermoed so. Dit gaan egter dalk te duur wees om tot by gestadigde toestande te simuleer sonder goeie beginwaardes.
\end{itemize}
\end{frame}

\begin{frame}
\frametitle{Solvers, LMM}
\begin{itemize}
	\item Ek het begin om 'n ander variant van scipy se numeriese integrasie gebruik
		\begin{itemize}
			\item Daar is algoritmes spesifiek vir "stiff" probleme waarmee ek dink ons te doen het (Radau, BDF).
			\item "Adaptive time stepping" wat baie beter werk as RK of Newmark met vaste tydstappe. 
			\item Dit is vinniger ook.
			\item Minder sensitief vir proporsionele demping.
		\end{itemize}
	\item Het tyd spandeer daaraan om DE vinniger te laat hardloop K(t) was relatief duur en met die numeriese integrasie word dit 'n probleem. Gebuik ook nou 'n smooth-square wave.
	\begin{itemize}
		\item +- 10 keer vinniger deur versigtig met konstantes te werk en hulle te hergebruik. 
		\item Sonder die vinniger kode sal optimeringsprobleem bykans onmoontlik gewees het.
	\end{itemize}
	\item Het komponente geweeg en kan nou die ware Jakob simuleer.
	\item Nog steeds nie heeltemal seker oor aandrywing van model nie. Het nog nie motor model implementeer nie.
		\begin{itemize}
		\item Translasie vryheidsgrade maak redelik sin.
		\item Rotasie vryheidsgrade maak my deurmekaar.
	\end{itemize}
\end{itemize}
\end{frame}


\begin{frame}
\frametitle{Admin}
\begin{itemize}
	\item Kampus toegang
	\item Wolfram skool 28 Junie - 17 Julie

\end{itemize}
\end{frame}

\end{document}