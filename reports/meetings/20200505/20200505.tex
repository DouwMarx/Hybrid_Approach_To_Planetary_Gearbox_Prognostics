\documentclass{beamer}

\usepackage[utf8]{inputenc}
\usepackage{amsmath}

\newcommand{\norm}[1]{\lVert#1\rVert}


%Information to be included in the title page:
\title{Sample title}
\author{Anonymous}
\institute{Overleaf}
\date{2014}



\begin{document}

\begin{frame}
\frametitle{Agenda}
\begin{itemize}
	\item System identification probleem en vereenvoudiging
	\item TSA
	\item Seinprosessering vooraf
	\item Solvers
	\item LMM
	\item Admin
\end{itemize}
\end{frame}

\begin{frame}
\frametitle{System identification probleem}
\begin{itemize}
	\item  Grey-box model
	\item Ons het te make met 'n baie nie-lineêre stelsel en mens sal waarskynlik 'n sequential monte carlo metode moet gebruik. 
	\begin{itemize}
		\item Particle Metropolis-Hastings of Particle Gibs sampling siende dat die data-likelihood nie met 'n Kalman filter berekenbaar is nie. 
	\end{itemize}
	\item Het aanvanklik direkte optimering gebruik (random init BFGS, genetic algorithm) op 'n model met 3 vryheidsgrade. 
	\begin{itemize}
		\item Gevorderdste probleem wat ek kom oplos op hierdie stadium is om die styfheid-verandering $\Delta K$ te benader (7 onbekendes) 
	\end{itemize}
	\item Het Metropolis-Hastings ook die probleem probeer aanpak met gebruik van pymc3 maar vir meer vryheidsgrade is daar twee pakkette in R wat dalk beter sal werk.
\end{itemize}
\end{frame}

\begin{frame}
\frametitle{Vereenvoudiging voorstel}
\begin{itemize}
	\item  Werk met delta-styfheid
	\item  Maak die inverse probleem vir die differensiaal vergelyking geraas vry: Gebruik direkte optimering.
	\item Maak gebruik van features vir die differensiaalvergelyking. Dit kan dalk toelaat dat die aanvangswaardes nie onbekendes is in die optimeringsprobleem nie.
	\item Korrigeer die fout wat die nie-stochastiese differensiaalvergelyking meebring met 'n data gedrewe model
	\item Laat die transmission-path model uit en vergelyk eerder die analitiese planeet vibrasie syn met die TSA. 
\end{itemize}
\end{frame}

\begin{frame}
\frametitle{Seinprosessering vooraf}
\begin{itemize}
	\item Is dit goeie praktyk om die frekwensies van die modale analise en bv die pomp waaier uit te filter?
	\item Maak dit sin om alle frekwensies bo en onder die lMM se frekwensies uit te filter? 
\end{itemize}
\end{frame}

\begin{frame}
\frametitle{Solvers}
\begin{itemize}
	\item Newmark-Beta werk vir 3 vryheidgrade (Daar was 'n tekenfout) maar sukkel met 12 vryheidsgrade as Beta1 en Beta2 nie reg gestel is nie.
	\item Newmark is vinniger by lae vryheidgrade maar net so vinnig soos Runge-Kutta met 12 vryheidsgrade
\end{itemize}
\end{frame}

\begin{frame}
\frametitle{LMM}
\begin{itemize}
	\item Toetse tot nou is gedoen met stilstaande ratkas
	\begin{itemize}
		\item Styfheid op planeet draer, Moment op sonrat
	\end{itemize}
	\item Plan is om moment op Son-rat te sit en die planeet draer teen konstante spoed te laat draai.
	\begin{itemize}
		\item Verwyder mens die planeet draer-vryheidsgrade van die model?
		\item Koppel mens 'n motor model?
	\end{itemize}
\end{itemize}
\end{frame}


\begin{frame}
\frametitle{Admin}
\begin{itemize}
	\item Rekenaar
	\item Universiteit toegang 
\end{itemize}
\end{frame}

\end{document}